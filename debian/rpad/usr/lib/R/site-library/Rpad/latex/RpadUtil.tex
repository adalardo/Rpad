\HeaderA{RpadUtil}{Rpad utilities}{RpadUtil}
\aliasA{RpadBaseFile}{RpadUtil}{RpadBaseFile}
\aliasA{RpadBaseURL}{RpadUtil}{RpadBaseURL}
\aliasA{RpadIsLocal}{RpadUtil}{RpadIsLocal}
\aliasA{RpadURL}{RpadUtil}{RpadURL}
\keyword{math}{RpadUtil}
\begin{Description}\relax
Rpad utilities to generate filenames or URL's
\end{Description}
\begin{Usage}
\begin{verbatim}
  RpadURL(filename = "")
  RpadBaseURL(filename = "")
  RpadBaseFile(filename = "")
  RpadIsLocal()
\end{verbatim}
\end{Usage}
\begin{Arguments}
\begin{ldescription}
\item[\code{filename}] the name of a file. 
\end{ldescription}
\end{Arguments}
\begin{Value}
\code{RpadURL} returns the URL for the given filename: "./filename" for
the local version of Rpad and "/Rpad/server/dd????????/filename" for
the server version. Use this to output HTML links for the user.

\code{RpadBaseURL} returns the base URL: "filename" for the local
version and "/Rpad/filename" for the client-server version. Use this
to point the user to data files or other links on the server that is
somewhere permanent. (The current R working directory is not
permanent in the client-server version.)

\code{RpadBaseFile} returns the file name relative to the base R
directory: "filename" for the local version and "../../filename" for
the client-server version. Use this in R to read in data files or save
data files somewhere permanent.
\end{Value}
\begin{Author}\relax
Tom Short, EPRI Solutions, Inc., (\email{tshort@eprisolutions.com})
\end{Author}
\begin{SeeAlso}\relax
\code{Rpad}, \code{RpadHTML}
\end{SeeAlso}
\begin{Examples}
\begin{ExampleCode}
  # make some data
  x <- 1:10
  y2 <- x^3
  save(x, y2, file = RpadBaseFile("testdata.RData"))
  # output a link to the user:
  HTMLon()
  cat("<a href='", RpadBaseURL("testdata.RData"), sep="")
  cat("'>Click</a> to download the test data.")
\end{ExampleCode}
\end{Examples}

